\documentclass[a4paper, 12pt]{article}

\usepackage{setspace}
\usepackage[margin=2cm]{geometry}
\usepackage{longtable}
\usepackage{hyperref}
\usepackage{enumitem}

%opening


\begin{document}

%\maketitle

\begin{titlepage}
\begin{doublespace}
\begin{center}
	\Huge\textbf{Title} 
\end{center}
\vspace{2cm} 
\begin{center} 
	\Large\textbf {Name:} \\[3pt]  
	Matr.-Nr.: \\
	Planspiel \\
	Tutor:  \\
	Technical University of Chemnitz \\
	Email: \\
	Winter semester 2025-2026 \\ [8cm] 
	\today
\end{center}
\end{doublespace}
\end{titlepage}



\section{Team Members}
\subsection{Juho Lee}
\textbf{Author: Juho Lee} \\

\subsection{Mariia Katsala}
\textbf{Author: Mariia Katsala} \\

\subsection{Muhammad Ibtisam Tanveer}
\textbf{Author: Muhammad Ibtisam Tanveer} \\

\subsection{Rocket Primm}
\textbf{Author: Rocket Primm} \\
Rocket is an American web developer pursing a master's degree at the TU Chemnitz. He did his bachelor in American at the University of North Carolina Wilmington, where he studied computer science. While he mainly focuses on web development, he is also interested in several areas of computer science such as AI, security, and general software development. He has the most experience with Javascript, Python, and Java, and uses NextJS, React, NodeJS, and more to create websites. He has previously worked on projects such as a Psychological Assessment Distribution and Analysis Service (PADAS), and is currently a work student of the TU Chemnitz. 

\subsection{Valeriia Bondareva}
\textbf{Author: Valeriia Bondareva} \\




\section{Company Culture}
\textbf{Author: Rocket Primm} \\
Creating the company culture for Devinche turned out to be more challenging than we first estimated. Our first challenge was simply understanding and differentiating the definitions of vision, mission, purpose, and values. Afterwards we had to consider whether these ideas should be applied to our company as a whole, or specifically to the project we chose. This proved especially challenging when trying to find our vision and purpose, which required us to look beyond the only project we will make, and into the future of a company that will only last the semester. Nonetheless, we feel confident that our company culture reflects us and how we will take on the development of our project.

\subsection{Vision}
Vision proved to be one of the more challenging topics to figure out. We were specifically struggling with the phrase "midterm 2-3 year strategy"\cite{planspiel_presentation_1}, since our project won't last that long. We kept thinking about it and decided some of the other topics, but still were not confident enough to solidify our vision. We were eventually able to come up with a vision after going back to the presentation and answering the questions "What do we want to be?"\cite{planspiel_presentation_1} and "How do we want to be seen by customers?"\cite{planspiel_presentation_1} for ourselves. We were then able to come up with the vision "We strive to be the leading platform for smart federated and AI systems". We believe this vision both aligns well with our chosen project, and allows for the opportunity of perhaps broadening our scope and moving beyond just one project.

\subsection{Mission}
We found the mission was one of the easier topics to figure out, and was the first topic we finalized. We wanted something short and memorable, basically a catch-phrase, as seen in many big company. The presentation already provided a few like Disney's "We make people happy"\cite{planspiel_presentation_1} and Zappos!'s "Delivering happiness"\cite{planspiel_presentation_1}, but we decided to look at a few more just to get more of an idea of how companies actually choose to express themselves so succinctly. Through this research, we discovered that there is no cookie cutter mold for a mission statement. Looking at some of the biggest companies in the world, such as Google, Amazon, and Netflix, did however show us different routes we could go down with our mission statement. Google has a very product focused mission statement, whereas Netflix made promises to their customers, suppliers, investors, and employees. Afterwards, we decided to keep our statement short and focused on our product. After some brainstorming, we decided on the phrase "Turning complexity into simplicity" With the possibility of a longer explanation as seen in some of the companies we researched. We believe this is a fitting mission statement since our project, Enfield, was made to make explaining complicated systems easier to understand for people not in the tech industry.

\subsection{Purpose}
As noted in the presentation, purpose was another tricky topic, leading to it being the last one that we decided on. We held tight to the explanations in the presentation, specifically the phrase "North Star"\cite{planspiel_presentation_1}, to help us come up with a purpose. We again focused on our project and came up with the purpose "To revolutionize how architects and developers visualize, document, and evolve distributed web applications". We believe that simplifying the diagramming process, both in creation and consumption, will lead to projects being built faster and better. We believe that this purpose statement accurately reflects that belief, as well as emphasize the impact we believe our product can make.

\subsection{Values}
We had to take some time to think about what values to choose. We again had the problem of figuring out how to choose the values given that our company started only a few weeks ago. Just like the mission statement, we looked at big companies to see what values they let guide them to success. At first some of us had the feeling that company values were generally meaningless positive characteristics that any company should hope to exemplify, but in reality we found relevant and focused values that we felt were accurate to the respective companies. This led us to think deeper and eventually come up with a whole list of values. The five values we came up with are as follows: Communication, creativity, simplicity, robustness, and innovation. We chose communication because that is the root of our product, communicating ideas. We wanted to centralize this point and let it influence our decisions, making it the first in the list. Creativity made the list because designing is inherently a creative process, even if it is with a architecture model that is surrounded by rules. Simplicity is also key since, as mentioned before, WAM diagrams are made in part for non-technical people to be able to understand complicated computer systems. We must consider simplicity in both the diagramming, as well as the user interface to ensure it is accessible for everyone. Robustness is important as it relates to the rules we must enforce in our modeling software. To ensure the simplicity and communication capabilities we want our product to have, we must make sure our models always follow the modeling rules. Our final value is motivation, which we feel must be considered for this product. We are creating a tool to model the real world, which will inevitably change. We therefore must change with it and improve where we must to stay relevant and useful. This value is further supported by one of our required tasks, adding AI elements to WAM, expanding it's previous modeling capabilities.



\section{Technologies}
\textbf{Author: Rocket Primm} \\
When deciding what tech stack we would use we considered what our group members have experience with, as well as what functionalities we believe will be required for our project. For the front/backend, we decided to use NextJS. Everyone in our group has experience with React, and two of us have experience with with NextJS itself. The two members with hands-on experience felt strongly about the simplicity of NextJS in comparison with using and Express server, so the other members agreed to learn it before beginning development. With NextJS we will easily be able to manage the many components, as well as the API routes we believe will be necessary for this project. We will also have full access to the many tools provided by React, which will help us in our search of a drawing/modeling framework that is not JointJS. In addition to the NextJS backend, we are considering having an additional python backend to help with the AI functionality requried in the project. Several of us have experience in this field and have all used python for it. We know that there are similar javascript libraries for AI, but simply due to experience we are considering it. For styling we chose TailwindCSS for its quick development, flexibility, and general familiarity with the group. We assume we will require many React components for the many components in WAM, and therefore thought it best to choose something that is more flexible and leads to less clutter than vanilla CSS. While TailwindCSS will be our main styling tool, we also discussed potentially using a component library like Material UI for further streamlined development. Finally, for our database we decided on MongoDB. We all immediately agreed that a non-relational database would be best for storing the diagrams. We also all have experience with it or another similar non-relational database making this an easy choice for us. Overall, we believe our choice of tools will not just be the best for the project, but for us as a team aswell.



\title{Monthly Meeting Notes}
\author{Report Author: Juho Lee}
\date{October 2025}

\maketitle

\section*{Summary of Meetings}
\begin{longtable}{|p{3cm}|p{2.5cm}|p{2.5cm}|p{6cm}|}
\hline
\textbf{Meeting Name} & \textbf{Date} & \textbf{Category} & \textbf{Agenda} \\
\hline
First Meeting & 14-Oct-25 & ad-hoc &raphical notation that provides a UML-like notation that is specifically tailored to the needs of inter-organizational web-base Build a team, decide on a communication tool. \\
\hline
Team Name & 17-Oct-25 & online & Decide on team name \& logo. \\
\hline
Team Logo & 19-Oct-25 & online & Finalize logo design. \\
\hline
Topic Decision & 21-Oct-25 & ad-hoc & Review topic presentations and set topic priority.\\
\hline
Website Design \& Build & 26-Oct-25 & online & Plan website structure, assign roles, configure tech stack. \\
\hline
Vision, Mission, Values \& Planning & 28-Oct-25 & regular & Define vision, mission, values; schedule weekly offline meetings; confirm tech stack. \\
\hline
First Topic Expert Meeting & 30-Oct-25 & topic expert & Clarify product requirements; define target users; discuss AI and cost considerations. \\
\hline
Vision, Mission, Report Content Finalization & 30-Oct-25 & regular & Assign report sections; set deadlines; finalize LaTeX setup and workflow. \\
\hline
\end{longtable}

\section*{Meeting Highlights}

\subsection*{First Meeting (Oct 14, 2025)}
\textbf{Agenda:} Build a team, decide on a communication tool.\\
\textbf{Discussion:} Team initially used WhatsApp; It was noted that having a linear conversation could reduce work efficiency over time, so it was decided to use a channel-based communication tool, such as Slack or Discord. The team preferred the discord as a result of familiarity.\\
\textbf{Decision:} Use Discord as main communication channel.

\subsection*{Team Name \& Logo (Oct 17 \& 19, 2025)}
\textbf{Agenda:} Decide on team name and logo.\\
\textbf{Discussion:} Starting with the name Webinch (Web in Chemnitz), Name refined from Devinch (Development in Chemnitz) to \textbf{Devinche}, referencing Leonardo Da Vinch. Logo options discussed; final design chosen emphasizing typography and aesthetic improvements.\\
\textbf{Decision:} Team name and logo finalized.

\subsection*{Topic Decision (Oct 21, 2025)}
\textbf{Agenda:} Review topic expert presentations and prioritize topics.\\
\textbf{Discussion:} Priorities were adjusted after the presentations, and the team showed interest in developing a web-based diagram editor. The initial priority order was Across → Chatbots → Enfield → WebXR, but after reviewing the experts' presentations, it was updated to Enfield → Across → Chatbots and WebXR.\\
\textbf{Decision:} Topic priority: Enfield → Across → Chatbots, WebXR.\\
\textbf{Action Items:} Valeria select Enfield at 10:00 AM; Prepare detailed plan(team).

\subsection*{Website Design \& Build (Oct 26, 2025)}
\textbf{Agenda:} Plan website design and structure, assign roles.\\
\textbf{Discussion:} Figma template created; tech stack selected (React, TypeScript, Next.js).\\
\textbf{Decision:} All members participate; Ibtisam to configure initial setup; initial website completion as soon as possible.

\subsection*{Vision, Mission, Values \& Planning (Oct 28, 2025)}
\textbf{Agenda:} Define vision, mission, values; confirm tech stack; schedule meetings; plan reports.\\
\textbf{Discussion:} The team discussed the project’s vision, mission, and values. Rocket shared preliminary definitions, and Rocket and Valeria suggested using a Da Vinci quote or artwork for inspiration. Juho and Rocket proposed the mission statement, *“Turning complexity into simplicity”*, which was agreed upon by all members. The vision was refined with ideas, and set as Creativity, Simplicity, and Robustness. Ibtisam proposed the technology stack: React and Next.js with TailwindCSS, Python (for AI if needed), and MongoDB, which the team accepted. The team also agreed to hold two fixed weekly offline meetings and share progress via LinkedIn and the Devinche Instagram account.\\
\textbf{Decision:} Mission: "Turning complexity into simplicity"; Values: Creativity, Simplicity, Robustness; Tech stack confirmed; project updates via social media; Valeria to set social media accounts and write blog posts.

\subsection*{First Topic Expert Meeting (Oct 30, 2025)}
\textbf{Agenda:} Re-explain topic, Q\&A.\\
\textbf{Discussion:} The team revisited the project requirements, focusing on the WAM editor. The editor should allow clients to draw WAM diagrams while existing libraries such as JointJS cannot be used. The system should support additional AI WAM elements in the future, initially designed for federated web and extendable to represent AI. Users should be able to create diagrams themselves, with an AI assistant also capable of generating them. Exported RDF diagrams must represent all relationships, and the focus is on system engineers who need to plan and understand system architecture while keeping it understandable for non-technical stakeholders. Security aspects are considered to maintain trust. Cost considerations include displaying AI-related operations and estimated additional costs, though precise estimation is not required at this stage.\\
\textbf{Action Items:} Research JointJS alternatives; plan RDF and AI integration; cost estimation methods.

\subsection*{Vision, Mission, Report Content Finalization (Oct 30, 2025)}
\textbf{Agenda:} Finalize report scope and responsibilities.\\
\textbf{Discussion:} LaTeX was chosen as the platform for collaborative report writing, allowing all team members to contribute efficiently through the Planspiel GitLab repository. The team confirmed the adoption of a SCRUM workflow with two-week sprints to manage tasks and ensure steady progress. Responsibilities for each section of the report were clearly assigned to individual members, with deadlines set to maintain accountability.\\
\textbf{Decision:} Report structure finalized; team members to complete sections by Nov 4; favicon design(Valeria); LaTeX Setup(Mariia); Update website with submitted content(Ibtisam);

\bibliographystyle{plain}
\bibliography{bibfile.bib}


\end{document}
